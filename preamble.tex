\usepackage{booktabs}
\pagestyle{headings}
\usepackage{fancyhdr} 
\usepackage{color}
\usepackage{polski}
\usepackage{float}
%\usepackage{amssymb, amsfonts, amsmath, bbold, stmaryrd} %pisanie matematyczne
  \makeatletter
	\renewcommand{\slimits@}{\limits} %niech inline limesy będą pod, nie obok
	\renewcommand{\nmlimits@}{\limits}
	\makeatother
	\renewcommand{\le}{\leqslant} %źle renderuje nierówności
	\renewcommand{\ge}{\geqslant}
\usepackage{mathtools} %\coloneqq

%technikalia
% czcionka Calibri
\usepackage{fontspec}
\setmainfont{Calibri}
% interlinia 1.5
\usepackage{setspace}
\setstretch{1.5}
% wcięcia akapitu 0.5cm
\setlength\parindent{0.5cm}
\setlength{\headheight}{15pt}
\usepackage{csquotes}
\usepackage{hyperref}
\usepackage[style=apa, backend=biber, alldates=year]{biblatex}
\addbibresource{book.bib}
\renewcommand\cite{\parencite}
\renewcommand\&{\bibstring{and}}\renewcommand\&{\bibstring{and}}
\DeclareCiteCommand{\citepage}[\mkbibparens]
  {\usebibmacro{cite:init}}
  {\usebibmacro{cite:post}}
  {}
  {\usebibmacro{postnote}}

%szybkie pisanie latex
\newcommand{\doubleC}{\mathbb{C}}
\newcommand{\doubleE}{\mathbb{E}}
\newcommand{\doubleK}{\mathbb{K}}
\newcommand{\doubleN}{\mathbb{N}}
\newcommand{\doubleP}{\mathbb{P}}
\newcommand{\doubleQ}{\mathbb{Q}}
\newcommand{\doubleR}{\mathbb{R}}
\newcommand{\doubleS}{\mathbb{S}}
\newcommand{\doubleZ}{\mathbb{Z}}
\newcommand{\scriptC}{\mathcal{C}}
\DeclareMathOperator{\tr}{tr}
\DeclareMathOperator{\Var}{Var}
\DeclareMathOperator{\Cov}{Cov}
% % nawiasy
\newcommand{\nz}[1]{\left( #1 \right)} %nawias zwykły
\newcommand{\nk}[1]{\left[ #1 \right]} %nawias kanciaty
\newcommand{\set}[1]{\left\{ #1 \right\}}
\newcommand{\abs}[1]{\left| #1 \right|}
\newcommand{\norm}[1]{\left\| #1 \right\|}
\newcommand{\floor}[1]{\left\lfloor #1 \right\rfloor}
\newcommand{\ceil}[1]{\left\lceil #1 \right\rceil}
% % ładne kwantyfikatory
%\newcommand{\A}[1]{\:\underset{#1}{\forall}\:}
%\newcommand{\E}[1]{\:\underset{#1}{\exists}\:}
\newcommand{\A}[1]{\bigwedge_{#1}}
\newcommand{\E}[1]{\bigvee_{#1}}
% % szczałki
\newcommand{\pcg}{\quad\Rightarrow\quad}%pociąga
\newcommand{\pcgc}{\quad&\Rightarrow\quad}%pociąga, centrując
\newcommand{\pcgk}{\Rightarrow}%pociąga krótko
\newcommand{\pcgd}{\quad\Longrightarrow\quad}%pociąga długo/daleko
\newcommand{\pcgt}[1]{\overset{#1}{\pcg}}%pociąga z tekstem
\newcommand{\wtw}{\quad\Leftrightarrow\quad}%wtedy i tylko wtedy
\newcommand{\wtwc}{\quad&\Leftrightarrow\quad}
\newcommand{\wtwk}{\Leftrightarrow}
\newcommand{\wtwd}{\Longleftrightarrow}
\newcommand{\xwtw}[1]{\overset{#1}{\wtw}}
\newcommand{\lwtw}{\leftrightarrow}%logiczne wtw
\makeatletter
\newcommand{\xto}[2][]{\ext@arrow 0099\rightarrowfill@{#1}{#2}}
\makeatother
\newcommand{\bbar}[1]{\overline{\overline{#1}}}
\newcommand{\whichis}[2]{\underbrace{#2}_{#1}}
\newcommand{\oo}{\infty}

\newcommand{\AAA}[1]{{\color{red}{\textbf{[#1]}}}}

% twierdzenia i dowody
\usepackage{amsthm}
\usepackage{pgf} % żeby TikZ działał
\usepackage[framemethod=TikZ]{mdframed} % ładne ramki
\newtheoremstyle{definicja}%
	{3pt}{3pt}{\itshape}{}
	{\bfseries}% Theorem head font
	{.}% Punctuation after theorem head
	{.5em}{}
\theoremstyle{definicja}
\newtheorem{defi}{Definicja}[chapter]
\surroundwithmdframed[roundcorner=5pt,skipabove=3mm,skipbelow=3mm]{defi}

\newtheoremstyle{twierdzenie}%
	{3pt}{3pt}{\itshape}{}
	{\bfseries}% Theorem head font
	{.}% Punctuation after theorem head
	{.5em}{}
\theoremstyle{twierdzenie}
\newtheorem{tw}{Twierdzenie}[chapter]
\surroundwithmdframed[roundcorner=5pt,skipabove=3mm,skipbelow=3mm]{tw}

\newtheoremstyle{lemat}%
	{3pt}{3pt}{\itshape}{}
	{\bfseries}% Theorem head font
	{.}% Punctuation after theorem head
 	{.5em}{}
\theoremstyle{lemat}
\newtheorem{lm}{Lemat}[chapter]
\surroundwithmdframed[roundcorner=5pt,skipabove=3mm,skipbelow=3mm]{lm}

\makeatletter % dowód
\renewenvironment{proof}[1][\proofname]{\par
  \pushQED{\qed}%
  \normalfont \topsep6\p@\@plus6\p@\relax
  \list{}{%
    \settowidth{\leftmargin}{\itshape\proofname:\hskip\labelsep}%
    \setlength{\labelwidth}{0pt}%
    \setlength{\itemindent}{-\leftmargin}%
  }%
  \item[\hskip\labelsep\itshape#1\@addpunct{:}]\ignorespaces
}{%
  \popQED\endlist\@endpefalse
}
\makeatother
\renewcommand\qedsymbol{$\blacksquare$} % znak końca dowodu